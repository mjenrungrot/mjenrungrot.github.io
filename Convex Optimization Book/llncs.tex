% This is LLNCS.DEM the demonstration file of
% the LaTeX macro package from Springer-Verlag
% for Lecture Notes in Computer Science,
% version 2.4 for LaTeX2e as of 16. April 2010
%
\documentclass{llncs}
%
\usepackage{fullpage}
\usepackage{amsmath}
\usepackage{amssymb}
\usepackage{makeidx}  % allows for indexgeneration
%

\newcommand{\pn}[1]{\left(#1\right)}

\begin{document}
\chapter*{Convex Sets}

\section{Affine and convex sets}
\subsection{Lines and line segments}
Suppose $x_1 \ne x_2$ are the points in $\mathbb{R}^n$. Points of the form
\begin{equation}
    y = \theta x_1 + (1-\theta) x_2,
\end{equation}
where $\theta \in \mathbb{R}$, are the \textit{line} passing through $x_1$ and $x_2$.

\subsection{Affine sets}
A set $C \subseteq \mathbb{R}^n$ is an \textit{affine} if the line through any two distinct points in $C$ lies in $C$, i.e., if $x_1$, $x_2 \in C$ and $\theta \in \mathbb{R}$, we have that $\theta x_1 + (1-\theta) x_2 \in C$.

Generalizing this idea to more than two points, we refer to a point of the form $\theta_1 x_1 + \ldots + \theta_k x_k$ where $\sum_{i=1}^{k}\theta_k = 1$ as an \textit{affine combination} of the points $x_1, \ldots, x_k$. 

\begin{definition}
An affine hull of $C$, denoted as $\textup{\textbf{aff} } C$, is the set of all affine combinations of points in some set $C \subseteq \mathbb{R}^n$:
\begin{equation}
    \textup{\textbf{aff} } C = \{\theta_1 x_1 + \cdots + \theta_k x_k \mid x_1, \cdots, x_k \in C, \theta_1 + \cdots + \theta_k = 1\}.
\end{equation}
\end{definition}

\subsection{Affine dimension and relative interior}




\section*{Problem Sets}
\begin{problem}[Distance Between Hyperplanes]
What is the distance between the two parallel hyperplanes $\{x \in \mathbb{R}^n \mid a^\top x = b_1\}$ and $\{ x \in \mathbb{R}^n \mid a^\top x = b_2\}$?
\end{problem}
\begin{solution}
Let $x_1$ and $x_2$ denote points where a vector $a$, which is orthogonal to the given two planes, intersects the first and the second hyperplanes. That is, we have
\[
    x_1 = \frac{b_1}{\|a\|^2}a ~~~~\text{and}~~~~ x_2 = \frac{b_2}{\|a\|^2}a.
\] 
Hence, we can simply see that the distance between two points is given by 
\[
    \|x_1 - x_2\| =  \frac{|b_1a - b_2a|}{\|a\|^2} = \frac{|b_1 - b_2|}{\|a\|}. 
\]
\end{solution}

\begin{problem}[Voronoi Description of a Halfspace]
Let $a$ and $b$ be distinct points in $\mathbb{R}^n$ and consider the set of points that are closer (in Euclidean norm) to $a$ than $b$, i.e., $\mathcal{C} = \{ x \mid \|x - a \|_2 \le \| x - b \|_2 \}$. The set $\mathcal{C}$ is a halfspace. Describe it explicitly as an inequality of the form $c^\top x \le d$. Draw a picture.
\end{problem}
\begin{solution}
In Euclidean norm, we can rewrite the inequality $\|x-a\|_2 \le \|x - b\|_2$ as:
\begin{align*}
    \pn{x - a}^\top \pn{x - a} \le \pn{x - b}^\top \pn{x - b} \\
    x^\top x - 2a^\top x + a^\top a \le x^\top x - 2b^\top x + b^\top b \\
    \pn{2b^\top - 2a^\top} x  \le \pn{b^\top b - a^\top a} 
\end{align*}
By letting $c = 2b - 2a$ and $d = b^\top b - a^\top a$, we can simply see that we reach at the following inequality:
\[
    c^\top x \le d
\]
which explains that $\mathcal{C}$ is a halfspace. In geometric point of view, consider the points that have the same distance to points $a$ and $b$. We notice that the set of these points has to be normal to the directional vector $b - a$. 
\end{solution}

\begin{problem}[Common Convex Sets]
Which of the following sets is convex?
\begin{enumerate}
    \item A slab, i.e., a set of the form $\{x \in \mathbb{R}^n \mid \alpha \le a^\top x \le \beta\}$
    \item A rectangle, i.e., a set of the form $\{ x \in \mathbb{R}^n \mid \alpha_i \le x_i \le \beta_i, i = 1, \ldots, n\}$. A rectangle is sometimes called a hyperrectangle when $n > 2$.
    \item A wedge, i.e., $\{x \in \mathbb{R}^n | a_1^\top x \le b_1, a_2^\top x \le b_2\}$
    \item The set of points closer to a given point than a given set, i.e., $\{ x \mid \|x - x_0\|_2 \le \|x - y \|_2$ for all $y \in S\}$ where $S \subseteq \mathbb{R}^n$
    \item The set of points closer to one set than another, i.e., $\{ x \mid \textbf{dist}(x, S) \le \textbf{dist}(x, T)\}$, where $S,T \subseteq \mathbb{R}^n$, and $\textbf{dist}(x,S) = \inf\{\|x - z\|_2 \mid z \in S\}$
\end{enumerate}
\end{problem}
\begin{solution} We have that 
\begin{enumerate}
    \item Consider the points $x_1, x_2 \in \{x \in \mathbb{R}^n \mid \alpha \le a^\top x \le \beta\}$. We have that
    \begin{align*}
    		\alpha \le a^\top x_1 \le \beta  ~~~~\text{and}~~~~ \alpha \le a^\top x_2 \le \beta 
    \end{align*}
    Given $\theta \in \mathbb{R}$ such that $0 \le \theta \le 1$, considering the point $x = \theta x_1 + \pn{1 - \theta}x_2$, we have
    \begin{center}
    \begin{align*}
    		\theta\alpha + \pn{1 - \theta}\alpha \le \theta\pn{\alpha^\top x_1} + \pn{1 - \theta}\pn{\alpha^\top x_2} \le \theta\beta + \pn{1 - \theta}\beta \\
    		\alpha \le \alpha^\top \pn{\theta x_1 + \pn{1-\theta}x_2} \le \beta \\
    		\alpha \le \alpha^\top x \le \beta 
			\end{align*}
    	\end{center}
    	which indicates that $x \in \{x \in \mathbb{R}^n \mid \alpha \le a^\top x \le \beta\}$. In other words, a slab is convex.
    \item Consider the points $x_1, x_2 \in \{x \in \mathbb{R}^n \mid \alpha_i \le x_i \le \beta_i, i = 1, \ldots, n\}$. We have that
			\[
					\alpha_i \le x_{1,i} \le \beta_i ~~~~\text{and}~~~~ \alpha_i \le x_{2,i} \le \beta_i 
			\]
			for all $i \in \{1,2,\ldots,n\}$. Considering a convex combination for $0 \le \theta \le 1$ between $x_1$ and $x_2$, i.e. $y = \theta x_1 + \pn{1-\theta}x_2$, we have
			\begin{align*}
				\theta \alpha_i + \pn{1-\theta} \alpha_i \le \theta x_{1,i} + \pn{1-\theta}x_{2,i} \le \theta \beta_i + \pn{1 - \theta}\beta_i \\
				\alpha_i \le y_i \le \beta_i
			\end{align*}
			Hence, $y$ is in the set of a rectangle. Therefore, a rectangle is convex.
			\item N/A
			\item N/A
			\item N/A
\end{enumerate}
\end{solution}

\begin{problem}[Some Sets of Probability Distributions]
Let $x$ be a real-valued random variable with $\textbf{prob}(x = a_i) = p_i$, $i = 1, \ldots, n$, where $a_1 < a_2 < \cdots < a_n$. Of course $p \in \mathbb{R}^n$ lies in the standard probability simplex $P = \{ p \mid \mathbf{1}^\top p = 1, p \succeq 0\}$. Which of the following conditions are convex in $p$? (That is, for which of the following conditions is the set of $p \in P$ that satisfy the condition convex?)
\end{problem}
\begin{solution}
N/A
\end{solution}

%
% ---- Bibliography ----
%
\begin{thebibliography}{}
% %
% \bibitem[1980]{2clar:eke}
% Clarke, F., Ekeland, I.:
% Nonlinear oscillations and
% boundary-value problems for Hamiltonian systems.
% Arch. Rat. Mech. Anal. 78, 315--333 (1982)

% \bibitem[1981]{2clar:eke:2}
% Clarke, F., Ekeland, I.:
% Solutions p\'{e}riodiques, du
% p\'{e}riode donn\'{e}e, des \'{e}quations hamiltoniennes.
% Note CRAS Paris 287, 1013--1015 (1978)

% \bibitem[1982]{2mich:tar}
% Michalek, R., Tarantello, G.:
% Subharmonic solutions with prescribed minimal
% period for nonautonomous Hamiltonian systems.
% J. Diff. Eq. 72, 28--55 (1988)

% \bibitem[1983]{2tar}
% Tarantello, G.:
% Subharmonic solutions for Hamiltonian
% systems via a $\bbbz_{p}$ pseudoindex theory.
% Annali di Matematica Pura (to appear)

% \bibitem[1985]{2rab}
% Rabinowitz, P.:
% On subharmonic solutions of a Hamiltonian system.
% Comm. Pure Appl. Math. 33, 609--633 (1980)

\end{thebibliography}
\clearpage
\addtocmark[2]{Author Index} % additional numbered TOC entry
\renewcommand{\indexname}{Author Index}
\printindex
\clearpage
\addtocmark[2]{Subject Index} % additional numbered TOC entry
\markboth{Subject Index}{Subject Index}
\renewcommand{\indexname}{Subject Index}
%%                                                           clmomu01.ind
%-----------------------------------------------------------------------
% CLMoMu01 1.0: LaTeX style files for books
% Sample index file for User's guide
% (c) Springer-Verlag HD
%-----------------------------------------------------------------------
\begin{theindex}
\item Absorption\idxquad 327
\item Absorption of radiation \idxquad 289--292,\, 299,\,300
\item Actinides \idxquad 244
\item Aharonov-Bohm effect\idxquad 142--146
\item Angular momentum\idxquad 101--112
\subitem algebraic treatment\idxquad 391--396
\item Angular momentum addition\idxquad 185--193
\item Angular momentum commutation relations\idxquad 101
\item Angular momentum quantization\idxquad 9--10,\,104--106
\item Angular momentum states\idxquad 107,\,321,\,391--396
\item Antiquark\idxquad 83
\item $\alpha$-rays\idxquad 101--103
\item Atomic theory\idxquad 8--10,\,219--249,\,327
\item Average value\newline ({\it see also\/} Expectation value)
15--16,\,25,\,34,\,37,\,357
\indexspace
\item Baker-Hausdorff formula\idxquad 23
\item Balmer formula\idxquad 8
\item Balmer series\idxquad 125
\item Baryon\idxquad 220,\,224
\item Basis\idxquad 98
\item Basis system\idxquad 164,\,376
\item Bell inequality\idxquad 379--381,\,382
\item Bessel functions\idxquad 201,\,313,\,337
\subitem spherical\idxquad 304--306,\, 309,\, 313--314,\,322
\item Bound state\idxquad 73--74,\,78--79,\,116--118,\,202,\, 267,\,
273,\,306,\,348,\,351
\item Boundary conditions\idxquad 59,\, 70
\item Bra\idxquad 159
\item Breit-Wigner formula\idxquad 80,\,84,\,332
\item Brillouin-Wigner perturbation theory\idxquad 203
\indexspace
\item Cathode rays\idxquad 8
\item Causality\idxquad 357--359
\item Center-of-mass frame\idxquad 232,\,274,\,338
\item Central potential\idxquad 113--135,\,303--314
\item Centrifugal potential\idxquad 115--116,\,323
\item Characteristic function\idxquad 33
\item Clebsch-Gordan coefficients\idxquad 191--193
\item Cold emission\idxquad 88
\item Combination principle, Ritz's\idxquad 124
\item Commutation relations\idxquad 27,\,44,\,353,\,391
\item Commutator\idxquad 21--22,\,27,\,44,\,344
\item Compatibility of measurements\idxquad 99
\item Complete orthonormal set\idxquad 31,\,40,\,160,\,360
\item Complete orthonormal system, {\it see}\newline
Complete orthonormal set
\item Complete set of observables, {\it see\/} Complete
set of operators
\indexspace
\item Eigenfunction\idxquad 34,\,46,\,344--346
\subitem radial\idxquad 321
\subsubitem calculation\idxquad 322--324
\item EPR argument\idxquad 377--378
\item Exchange term\idxquad 228,\,231,\,237,\,241,\,268,\,272
\indexspace
\item $f$-sum rule\idxquad 302
\item Fermi energy\idxquad 223
\indexspace
\item H$^+_2$ molecule\idxquad 26
\item Half-life\idxquad 65
\item Holzwarth energies\idxquad 68
\end{theindex}

\end{document}
